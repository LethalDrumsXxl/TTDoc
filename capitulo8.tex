\chapter{Implantaci�n}
\label{implantacion}

En el presenta cap�tulo, se describen las tecnolog�as y herramientas necesarias para la correcta implantaci�n del sistema desarrollado. 

\section{Infraestructura}
\label{infraestructura}

Para la implantaci�n del sistema, se cuenta con un servidor facilitado por la Escuela de Ingenier�a Civil Inform�tica, el cual tiene las siguientes caracter�sticas y softwares necesarios para el correcto funcionamiento del sistema: \\

\begin{itemize}

\item \textbf{Procesador:} Intel Xeon de 2.00 GHz. \\
\item \textbf{Memoria Ram:} 4GB. \\
\item \textbf{Sistema Operativo:} Microsoft Windows Server 2008 Enterprise. \\
\item \textbf{Servidor de Aplicaci�n:} Apache Tomcat 7.0.55. \\
\item \textbf{Motor de Base de Datos:} PostgreSQL 9.3. \\

\end{itemize}

\section{Actividades Previas a la Implantaci�n}
\label{actividadespreviasalaimplantacion}

En esta secci�n se describe las configuraciones previas, hechas en el servidor, para el correcto funcionamiento del sistema. \\

\subsection{Configuraci�n del Motor de Base de Datos}
\label{configuracionpsql}

Antes de implantar el sistema, se hizo una previa configuraci�n de PostgreSQL\footnote{\url{http://www.postgresql.org/}}, el cual ya estaba instalado en el servidor. Para ello se siguieron los siguientes pasos: \\ 

\begin{enumerate}

\item Se cre� la base de datos \textit{GCAR}, que es la base de datos con la que trabaja el sistema. \\
\item Se cre� el usuario \textit{GCAR} y se le dieron los permisos necesarios para administrar la base de datos  del sistema construido. \\
\item Utilizando el nuevo usuario creado, se crearon las tablas necesarias para el correcto funcionamiento del sistema. \\

\end{enumerate}

Luego de estos pasos, el servidor se encuentra listo para la implantaci�n del sistema. \\

\section{Actividades para la Implantaci�n}
\label{actividadesparalaimplantacion}

En esta secci�n se describen los pasos de la instalaci�n del sistema en el servidor de aplicaciones. \\

\subsection{Instalaci�n del Sistema en el Servidor de Aplicaciones}
\label{instalacion}

El servidor de aplicaciones utilizado es Apache Tomcat\footnote{\url{http://tomcat.apache.org/}}, el cual se encontraba previamente instalado en el servidor. A continuaci�n, se listan los pasos realizados durante la implantaci�n del sistema: \\

\begin{enumerate}

\item El archivo \textit{.war}, correspondiente al sistema completo, fue instalado en el directorio \texttt{C:/apache-tomcat-7.0.55/webapps}. Este directorio contiene todas las aplicaciones web que el servidor administra. \\
\item Se ingreso al directorio \texttt{C:/apache-tomcat-7.0.55/bin} y se ejecut� el archivo \texttt{startup.bat}. De esta manera, el servidor compila el archivo \textit{.war}, crea las carpetas necesarias y ejecuta el sistema. \\

\item Se ingres� a la p�gina web \texttt{http://localhost:8090/GCAR} para revisar si el servidor est� corriendo correctamente el sistema. \\

\end{enumerate}

Luego de esto, y con el servidor trabajado en perfectas condiciones, el sistema est� implantado en el servidor. De este modo, ya es posible trabajar con \textit{GCAR} y probar todas sus funcionalidades. \\

\section{Actividades Posteriores a la Implantaci�n}
\label{actividaesposterioresalaimlantacion}

Ya con el sistema implantado, se hace entrega del Manual de Usuario al administrador del sistema. As�, el administrador podr� conocer todas y cada las funcionalidades del sistema y el correcto uso de ellas. \\

El manual explica detalladamente los pasos a seguir para llevar a cabo alguna tarea espec�fica. Adem�s est� separado por secciones, las cuales atienden por separado las funcionalidades de \textit{GCAR}. Cada explicaci�n agrega una o m�s im�genes con el fin de que el aprendizaje sea m�s f�cil para el usuario. El manual de usuario se presenta en el Ap�ndice \ref{manualdeusuario}. \\
